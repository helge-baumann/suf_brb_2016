\section{Vom Rohdatensatz zum Scientific Use File}



\subsection{Missing Labels}\label{var_miss}

Die \glqq ungültigen\grqq\xspace bzw. fehlenden Werte (\textit{missing values}) des Rohdatensatzes wurden zwecks Nutzerfreundlichkeit überarbeitet. Zum einen wurden die bereits gelabelten Werte für \glqq Weiß nicht\grqq , \glqq verweigert\grqq\xspace und \glqq trifft nicht zu\grqq\xspace auf negative Werte gesetzt, um Berechnungen zu vereinfachen. Zum anderen wurden ungelabelte, fehlende Werte dahingehend differenziert, ob sie durch die \hyperref[langkurz]{Fragebogenversion} oder eine etwaige Filterführung fehlen (die Filterführungen können durch den Fragebogen nachvollzogen werden). Folgende Transformationen wurden vorgenommen:

\begin{itemize}
	\item \glqq Weiß nicht\grqq -Angaben:
	\begin{itemize}
	\item im Rohdatensatz 8, 98, 998, 9998, 999998, 999998 
	\item im Scientific Use File durchgehend -8
	\end{itemize}
	\item Verweigerungen:
	\begin{itemize}
	\item im Rohdatensatz 7, 97, 997, 9997, 999997, 999997 
	\item im Scientific Use File durchgehend -7
	\end{itemize}
	\item \glqq Trifft nicht zu\grqq -Angaben:
	\begin{itemize}
	\item im Rohdatensatz 6
	\item im Scientific Use File durchgehend -6
	\end{itemize}
	\item \glqq Missing durch Filterführung\grqq : -5
	\item \glqq Item in Fragebogenversion nicht erhoben\grqq : -4

\end{itemize}

\subsection{Generierte Variablen}\label{var_generiert} 

Im Folgenden wird der Erzeugungsprozess generierter Variablen beschrieben. Dabei kann im Einzelnen
nachvollzogen werden, durch welche Ursprungsvariablen bzw. welche Werte dieser Ursprungsvariablen die neuen, generierten
Variablen zustande kommen. Generiert wurden solche Variablen, die sich aus mehreren Fragen zusammensetzen. Dies betrifft einerseits Anteilswerte von Beschäftigtenstrukturangaben, die aus der Division von  der jeweiligen Beschäftigtenstruktur und der Betriebsgröße (Variable \hyperref[var:D1]{D1}) generiert wird. Dies betrifft andererseits die Struktur des Betriebsrates, für die die jeweils erhobenen Angaben (z.B., wie viele Frauen es im Betriebsrat gibt) durch die Größe des Betriebsrates (Variable \hyperref[var:M3]{M3}) dividiert wurden. Diese Variablen sind im Datensatz des Scientific Use Files durch das Suffix \glqq \_gen \grqq\xspace gekennzeichnet, auch ihr Variablenlabel enthält einen entsprechenden Hinweis (\grqq generierte Variable\grqq )

\begin{longtable}{!{\color{black}\vline width 1pt} L{2.5cm} !{\color{black}\vline width 1pt} L{9.125cm} | L{5.125cm} !{\color{black}\vline width 1pt}  }
	
	\toprule
		\textbf{Variable} & \textbf{WENN:} & \textbf{DANN:}  \\ 
	\midrule
	\endfirsthead
	
	\toprule
		\textbf{Variable} & \textbf{WENN:} & \textbf{DANN:}  \\ 
	\midrule
	\endhead
	
	\midrule
	
	\endfoot
	\bottomrule
	\endlastfoot
		
	\input{E1_Generierte_Variablen.tex}
	
\end{longtable}

\subsection{Vergröberte Variablen}\label{var_kategorisiert}

Einige Variablen des Rohdatensatzes enthalten einzelne Angaben, die die Anonymit?t des Betriebes bzw. des Betriebsrates gefährden. Aus diesen Variablen wurden klassierte Daten generiert. Diese Variablen sind im Datensatz des Scientific Use Files durch das Suffix \glqq \_kat \grqq\xspace gekennzeichnet, auch ihr Variablenlabel enthält einen entsprechenden Hinweis (\grqq vergröbert\grqq ). Dies betrifft im Einzelnen:

\begin{enumerate}

\item \hyperref[var:D1]{D1}: Anzahl der Beschäftigten im Betrieb 
\item \hyperref[var:I7]{I7}: Höhe des Krankenstandes am 01.12.2014 in Prozent
\item \hyperref[var:M1]{M1}: Existenzdauer des Betriebsrats in Jahren
\item \hyperref[var:M2]{M2}: Wahljahr des Betriebsrats
\item \hyperref[var:M3]{M3}: Anzahl der Betriebsratsmitglieder
\item \hyperref[var:M13]{M13}: Alter des Betriebsratsvorsitzenden
\item \hyperref[var:R1a]{R1a}: Anzahl der Besch?ftigten im Gesamtunternehmen

\end{enumerate}

\subsection{Gelöschte Variablen}\label{var_geloescht}

Dem Scientific Use File wurden gegenüber dem Rohdatensatz eine Reihe von Variablen entnommen. Damit wird zunächst die Anonymität der Zielpersonen gewährleistet, deren Angaben vor allem zur methodischen Qualit?tssicherung der Fragen erhoben werden. Da es sich jedoch um eine Gremienbefragung handelt, für die Zielperson lediglich als Repr?sentant fungiert, sind diese Angaben für die inhaltliche Forschung nicht notwendig. Auch die Entfernung offener Angaben dient, da sich darunter exakte Wortlaute finden, der Anonymisierung des Datensatzes. 

Weiter wurden Variablen zur Übersichtlichkeit des Datensatzes entnommen. Darunter fallen die Methodenvariablen (z.B. die Reihenfolge, in der gewisse Fragen eingespielt wurden), aber auch alle Variablen, die zu \hyperref[var_generiert]{generierten} oder \hyperref[var_kategorisiert]{kategorisierten} Variablen weiterverarbeitet wurden.

\begin{longtable}{!{\color{black}\vline width 1pt} L{2.5cm} !{\color{black}\vline width 1pt} L{2.5cm} | L{11.75cm} !{\color{black}\vline width 1pt}  }
	
	\toprule
	\textbf{Löschgrund} & \textbf{Variable} & \textbf{Variablenbezeichnung}  \\ 
	\midrule
	\endfirsthead

	\toprule
		\textbf{Löschgrund} & \textbf{Variable} & \textbf{Variablenbezeichnung}  \\ 
	\midrule
	\endhead
	
	\midrule
	
	\endfoot
	\bottomrule
	\endlastfoot
		
	\input{E2_Geloeschte_Variablen.tex}
	
\end{longtable}

\subsection{Ergänzte Variablen}\label{var_ergaenzt}

Dem Scientific Use File wurden vier Gewichtungsfaktoren sowie eine Interviewerkennung hinzugefügt. Die Herleitung der Gewichtungsfaktoren entnehmen Sie bitte dem Methodenbericht der WSI-Betriebsrätebefragung 2015. Dabei sollten Sie insbesondere ber?cksichtigen, dass Sie, je nachdem, ob Sie Fragen der Kurz- oder Langversion des Fragebogens auswerten, unterschiedliche Gewichtungsfaktoren verwenden müssen. Zudem ist zu beachten, dass den Gewichtungsfaktoren unterschiedliche Grundgesamtheiten zugrunde liegen.\footnote{Das WSI empfiehlt die Verwendung der Gewichtungsfaktoren für die Grundgesamtheit der Betriebe mit Betriebsrat.} Mithilfe der Interviewerkennung können Clusterungseffekte z.B. in Mehrebenenanalysen berücksichtigt werden. Die ergänzten Variablen und ihre Labels lauten:

\begin{enumerate}

\item \hyperref[var:suf:gewab:k]{gewab\_k}: Gewichtungsfaktor Grundgesamtheit aller Betriebe (Kurzversion)
\item \hyperref[var:suf:gewab:l]{gewab\_l}: Gewichtungsfaktor Grundgesamtheit aller Betriebe (Langversion)
\item \hyperref[var:suf:gewbr:k]{gewbr\_k}: Gewichtungsfaktor Grundgesamtheit Betriebe mit Betriebsrat (Kurzversion)
\item \hyperref[var:suf:gewbr:l]{gewbr\_l}: Gewichtungsfaktor Grundgesamtheit Betriebe mit Betriebsrat (Langversion)
\item \hyperref[var:suf:internr:n]{internr\_n}: Interviewernummer
\item \hyperref[var:suf:westost]{westost}: WESTOST
\item \hyperref[var:suf:branche10]{branche10}: Branche (WZ 2008)

\end{enumerate}
